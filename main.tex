\documentclass[bachelor,openright,notchinese]{sustcthesis}
% 默认twoside 双面打印
% 将master修改为bachelor, doctor or master
% 要使用adobe字体,添加adobefonts选项
% 使用euler数学字体,如不愿使用,去掉euler
% 使用外文写作,请添加notchinese

% 设置图形文件的搜索路径
\graphicspath{{figures/}}



%%%%%%%%%%%%%%%%%%%%%%%%%%%%%%
%% 封面部分
%%%%%%%%%%%%%%%%%%%%%%%%%%%%%%



 % 中文封面内容
  \title{六次循环域}%一般情况下扉页和封皮、书脊共用一个标题文本,可以不用定义\spinetitle(仅硕博有用), \covertitle(本硕博均有用)和\encovertitle(仅本科有用)。特殊情况见下。
  %特殊情况1:本例中\title命令里含有换行控制字符,这会导致制作书脊的时候出现错误,例如如果你注释掉\spinetitle{...}这一行就会报错。这时需要定义一个不含换行等命令的\spinetitle,这并不表示\spinetitle里不能有任何命令——只能使用有限的命令。
  %特殊情况2:本例中标题过长,所以需要缩小书脊标题的字号。
  %特殊情况3:本例中中英文混排,由于tex竖排的原理限制,中英文基线不重合,所以需要人工调整英文的基线。具体调整量根据不同字体有所不同。
  %\covertitle{六次循环域}
  %\covertitle{中文题目第一行\\中文题目第二行}
  %不要在此调整封皮字体大小! Do not set Cover Page font size here!
  %特殊情况4:本例中\title中含有多个换行,导致标题超过了两行。根据制本厂规定,封皮标题不能超过两行。因此需要定义封皮使用的标题\covertitle. 如果你注释掉这一行,就会发现封皮不符合规定。
  %\encovertitle{On Cyclic Sextic Fields}
  %\encovertitle{English Title Line 1\\English Title Line 2\\English Title Line 3}
  %不要在此调整封皮字体大小! Do not set Cover Page font size here!
  %特殊情况5:仅本科生有用。本科封皮中有英文标题,不超过三行。与上类似。
  \author{张 \ 文 \ 超}
  \depart{金融数学系}%系别,硕博请用系代号,本科请用全称如
  \major{数学专业}%专业,硕博请用全称,本科不需要
  \advisor{李景治\ 教授}
  \coadvisor{张贤科\ 教授,\ 余解台\ 教授}%第二导师,没有请注释掉
  \studentid{11110015}%For bachelor only
  \submitdate{二〇一四年九月}

  % 英文封面内容
  \entitle{On Cyclic Sextic Fields and Galois Extensions of Function Fields}
  \enauthor{Wenchao Zhang}
  \studentid{11110015}
  \endepart{Financial Mathematics}
  \enmajor{Mathematics}
  \enadvisor{Jingzhi Li}
  \encoadvisor{Xianke Zhang(Number Theory)}
  \encoadvisorsec{Jietai Yu(Galois Theory)}
  \ensubmitdate{September, 2014}
  
\begin{document}

% 封面 (By Taper: 学校封面已改,请在Word中填写好新封面,输出为pdf,之后插入文件开头即可。也可在Word中新封面后留白一页,以免打印时和诚信承诺书印于同一张纸。)
%\maketitle

%特别注意,以下述顺序为准,在对应部分添加文档部件,切勿颠倒顺序:
%本科论文的文档部件顺序是:
%    frontmatter:致谢、目录、中文摘要、英文摘要、
%    mainmatter: 正文章节
%    backmatter: 参考文献或资料注释、附录
%%%%%%%%%%%%%%%%%%%%%%%%%%%%%%
%% 前言部分
%%%%%%%%%%%%%%%%%%%%%%%%%%%%%%
\frontmatter
\makeatletter

\ifustc@bachelor
	%%%%%%%%%%%%%%%%%
	%本科论文修改这里
	%%%%%%%%%%%%%%%%%
	% 致谢
	\include{chapter/honest}
	% Preface is optional...
	\include{chapter/chap-preface}
	% 摘要
	\begin{center}
	{\erhao \ustc@entitle}
	
	{\sihao \ustc@enauthor}
	
	{(Department: \sihao \ustc@endepart, Advisor: \ustc@enadvisor)}
\end{center}

\begin{cnabstract}
本文主要分为两个部分,前半部分主要研究了六次循环域的结构。首先将代数数论的一般理论、二次域和循环三次域的主要理论和近期文献中有关三次、六次循环域的结果, 进行了整理综合。 在此基础上, 我们给出了一般六次循环域的整基,素分解的方法,具体给出了多个例子。例如,我们解决了最简单的复六次循环域——7次分圆域的判别式,整基,素分解。同时,利用一般的类数和单位的理论,计算了它的类数和单位群。其次,我们给出了一个实六次循环域,在解决该例子时先利用线性预解子的方法判定出所给多项式$f(x)=x^6-x^5-6x^4+6x^3+8x^2-8x+1$的伽罗瓦群为$C_6$,即得到其对应的分裂域为六次循环域。之后,通过六次循环域的结果给出该例子的二次子域和三次循环子域,从而进一步得到它的整基、素分解,并且我们还计算了这个域的类数。

六次循环域整基和素分解的一般结果是在利用解析数论方法得到判别式的前提下,利用其子域结构得出来的。

第二部分研究的是函数域上的伽罗瓦扩张。在经典的代数和代数几何理论中,L\"{u}roth定理揭示了一元函数域$K(x)$和基域$K$的中间域$E$是基域的单扩张(中间域不等于基域时为非代数扩张)。进一步地,本文假定$K(x)/E$为伽罗瓦扩张,不利用L\"{u}roth定理,证明了$E=K(u)$,其中$u\in K(x)$,可以被$\operatorname{Gal}(K(x)/E)$的初等对称多项式所确定。我们在其后也给出了一个经典$\operatorname{Gal}(K(x)/E)=D_3$的例子。

\keywords{伽罗瓦理论,六次循环域,整基,素分解,L\"{u}roth定理}
\end{cnabstract}

\begin{enabstract}
We have two parts in this thesis. In the first part, we study the structure of cyclic sextic field with many details on its subfields: quadratic field and cyclic cubic field. We reorganize main theory of algebraic number theory and some recent references on cyclic cubic field and cyclic sextic field. Based on these results, we solve the integral basis of the cyclic sextic field as well as find the prime decomposition algorithm. Precisely, we give some examples applying our results. For example, we solve discriminant, integral basis, prime decomposition of 7-cyclotomic field. We also compute its unit group and class number using general theory. We also give an example for real cyclic sextic field. Using linear resolvent method, we first determine that the minimal polynomial has Galois group $C_6$. Then we also find the quadratic subfield and cyclic cubic subfield in order to get the integral basis and prime decomposition. We compute the class number as well.

The general results for integral basis and prime decomposition of cyclic sextic field are given by structures of its subfields based on its discriminant which is computed through analytic number theory's method. 

In the second part, we discuss Galois extensions of a function field. In the classical theory of algebra and algebraic geometry, L\"{u}roth's theorem reveals that any intermediate field $E/K$ of $K(x)/K$ (where $x$ is transcendental extension over $K$) is a simple extension. More precisely, assuming that $K(x)/E$ is Galois, without using L\"{u}roth's theorem, we prove that $E=K(u)$, where $u\in K(x)$ can be determined by the elementary symmetric polynomials. We then give a classical example for $\operatorname{Gal}(K(x)/E)=D_3$.   

\enkeywords{Galois Theory,Cyclic Sextic Fields, Integral Basis, Prime Decomposition, L\"{u}roth's Theorem}
\end{enabstract}
%此文件中含有中英文摘要
	
	%目录部分
	%目录
	\tableofcontents
	%默认表格、插图、算法索引名称分别为“表格索引”、“插图索引”和“算法索引”
	%如果需要自行修改lot,lof,loa的名称,请定义
	%\ustclotname{...}
	%\ustclofname{...}
	%\ustcloaname{...}

	% 表格索引
	%\ustclot
	% 插图索引
	%\ustclof
	%算法索引 
	%如果需要使用算法环境并列出算法索引,请加入补充宏包。
	%\ustcloa
	
	% For list of notations
    \include{chapter/denotation}
    
\else
	%%%%%%%%%%%%%%%%%
	%硕博论文修改这里
	%%%%%%%%%%%%%%%%%
	% 摘要
	\begin{center}
	{\erhao \ustc@entitle}
	
	{\sihao \ustc@enauthor}
	
	{(Department: \sihao \ustc@endepart, Advisor: \ustc@enadvisor)}
\end{center}

\begin{cnabstract}
本文主要分为两个部分,前半部分主要研究了六次循环域的结构。首先将代数数论的一般理论、二次域和循环三次域的主要理论和近期文献中有关三次、六次循环域的结果, 进行了整理综合。 在此基础上, 我们给出了一般六次循环域的整基,素分解的方法,具体给出了多个例子。例如,我们解决了最简单的复六次循环域——7次分圆域的判别式,整基,素分解。同时,利用一般的类数和单位的理论,计算了它的类数和单位群。其次,我们给出了一个实六次循环域,在解决该例子时先利用线性预解子的方法判定出所给多项式$f(x)=x^6-x^5-6x^4+6x^3+8x^2-8x+1$的伽罗瓦群为$C_6$,即得到其对应的分裂域为六次循环域。之后,通过六次循环域的结果给出该例子的二次子域和三次循环子域,从而进一步得到它的整基、素分解,并且我们还计算了这个域的类数。

六次循环域整基和素分解的一般结果是在利用解析数论方法得到判别式的前提下,利用其子域结构得出来的。

第二部分研究的是函数域上的伽罗瓦扩张。在经典的代数和代数几何理论中,L\"{u}roth定理揭示了一元函数域$K(x)$和基域$K$的中间域$E$是基域的单扩张(中间域不等于基域时为非代数扩张)。进一步地,本文假定$K(x)/E$为伽罗瓦扩张,不利用L\"{u}roth定理,证明了$E=K(u)$,其中$u\in K(x)$,可以被$\operatorname{Gal}(K(x)/E)$的初等对称多项式所确定。我们在其后也给出了一个经典$\operatorname{Gal}(K(x)/E)=D_3$的例子。

\keywords{伽罗瓦理论,六次循环域,整基,素分解,L\"{u}roth定理}
\end{cnabstract}

\begin{enabstract}
We have two parts in this thesis. In the first part, we study the structure of cyclic sextic field with many details on its subfields: quadratic field and cyclic cubic field. We reorganize main theory of algebraic number theory and some recent references on cyclic cubic field and cyclic sextic field. Based on these results, we solve the integral basis of the cyclic sextic field as well as find the prime decomposition algorithm. Precisely, we give some examples applying our results. For example, we solve discriminant, integral basis, prime decomposition of 7-cyclotomic field. We also compute its unit group and class number using general theory. We also give an example for real cyclic sextic field. Using linear resolvent method, we first determine that the minimal polynomial has Galois group $C_6$. Then we also find the quadratic subfield and cyclic cubic subfield in order to get the integral basis and prime decomposition. We compute the class number as well.

The general results for integral basis and prime decomposition of cyclic sextic field are given by structures of its subfields based on its discriminant which is computed through analytic number theory's method. 

In the second part, we discuss Galois extensions of a function field. In the classical theory of algebra and algebraic geometry, L\"{u}roth's theorem reveals that any intermediate field $E/K$ of $K(x)/K$ (where $x$ is transcendental extension over $K$) is a simple extension. More precisely, assuming that $K(x)/E$ is Galois, without using L\"{u}roth's theorem, we prove that $E=K(u)$, where $u\in K(x)$ can be determined by the elementary symmetric polynomials. We then give a classical example for $\operatorname{Gal}(K(x)/E)=D_3$.   

\enkeywords{Galois Theory,Cyclic Sextic Fields, Integral Basis, Prime Decomposition, L\"{u}roth's Theorem}
\end{enabstract}
%此文件中含有中英文摘要
	% 目录
	\tableofcontents
	%默认表格、插图、算法索引名称分别为“表格索引”、“插图索引”和“算法索引”
	%如果需要自行修改lot,lof,loa的名称,请定义
	%\ustclotname{...}
	%\ustclofname{...}
	%\ustcloaname{...}

	% 表格索引
	\ustclot
	% 插图索引
	\ustclof
	%算法索引 
	%如果需要使用算法环境并列出算法索引,请加入补充宏包。
	%\ustcloa
	
	%符号说明,需要加入补充包
	\include{chapter/denotation}%不是必需的,如果不想列出请注释掉
\fi
\makeatother

%%%%%%%%%%%%%%%%%%%%%%%%%%%%%%
%% 正文部分
%%%%%%%%%%%%%%%%%%%%%%%%%%%%%%
\mainmatter
  \include{chapter/introduction}	
  %自行添加
  \include{chapter/chap-three}
  \include{chapter/chap-two}
  \include{chapter/chap-four} 
  \include{chapter/chap-five}
  \include{chapter/chap-six}
  \include{chapter/chap-seven}
  \include{chapter/chap-eight}
  \include{chapter/conclusion} 


%%%%%%%%%%%%%%%%%%%%%%%%%%%%%%
%% 附件部分
%%%%%%%%%%%%%%%%%%%%%%%%%%%%%%
\backmatter
  %结语
  
  % 参考文献
  % 使用 BibTeX
  % 选择参考文献的排版格式。注意ustcbib这个格式不保证完全符合要求,请自行决定是否使用
  \bibliographystyle{sustcbib}%{GBT7714-2005NLang-UTF8}
  \bibliography{bib/tex}
  \nocite{*} % for every item
  % 不使用 BibTeX
  % \include{chapter/bib}
  
  
  
  % 附录,没有请注释掉
  \begin{appendix}
  \include{chapter/chap-appD}
  \include{chapter/chap-appB}
  \include{chapter/chap-appA}
  
  
  \end{appendix}
   
  \include{chapter/thanks}

  

\end{document}
